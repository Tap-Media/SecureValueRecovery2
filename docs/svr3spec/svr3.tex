\documentclass{article}

% Language setting Replace `english' with e.g. `spanish' to change the document
% language
\usepackage[english]{babel}

% Set page size and margins Replace `letterpaper' with `a4paper' for UK/EU
% standard size
\usepackage[letterpaper,top=2cm,bottom=2cm,left=3cm,right=3cm,marginparwidth=1.75cm]{geometry}

% Useful packages
\usepackage{amsmath}
\usepackage{graphicx}
\usepackage[colorlinks=true, allcolors=blue]{hyperref}
\usepackage[ n, % or lambda
advantage, operators, sets, adversary, landau , probability, notions, logic, ff,
mm, primitives, events, complexity, oracles, asymptotics, keys]{cryptocode}
%% Primitives
\newcommand{\OPRF}{\pcalgostyle{OPRF}} 
\newcommand{\POPRF}{\pcalgostyle{POPRF}} 
\newcommand{\VOPRF}{\pcalgostyle{VOPRF}}

\newcommand{\Blind}{\pcalgostyle{Blind}}
\newcommand{\BlindEvaluate}{\pcalgostyle{BlindEvaluate}}
\newcommand{\BlindEvaluateForClient}{\pcalgostyle{BlindEvaluateForClient}}
\newcommand{\Finalize}{\pcalgostyle{Finalize}}

\newcommand{\PPSSStore}{\pcalgostyle{PPSSStore}}
\newcommand{\PPSSRecover}{\pcalgostyle{PPSSRecover}}

\newcommand{\ServerCreateOPRFVersion}{\pcalgostyle{ServerCreateOPRFVersion}}


%% Hashes
\newcommand{\HashToPoint}{\pcalgostyle{HashToPoint}}
\newcommand{\HashToScalar}{\pcalgostyle{HashToScalar}}
\newcommand{\HashToField}{\pcalgostyle{HashToField}}
\newcommand{\EncodeToField}{\pcalgostyle{EncodeToField}}

%% Variables
\newcommand{\oprfinput}{\pcalgostyle{oprf\_input}}
\newcommand{\oprfkeys}{\pcalgostyle{oprf\_keys}}
\newcommand{\usage}{\pcalgostyle{usage}}
\newcommand{\usagecount}{\pcalgostyle{usage\_count}}
\newcommand{\maxuses}{\pcalgostyle{max\_uses}}
\newcommand{\blind}{\pcalgostyle{blind}}
\newcommand{\blindedElement}{\pcalgostyle{blindedElement}}
\newcommand{\evaluatedElement}{\pcalgostyle{evaluatedElement}}

\newcommand{\clientstate}{\pcalgostyle{client\_state}}
\newcommand{\serverstate}{\pcalgostyle{server\_state}}
\newcommand{\clientid}{\pcalgostyle{client\_id}}
\newcommand{\client}{\pcalgostyle{client}}
\newcommand{\server}{\pcalgostyle{server}}
\newcommand{\servers}{\pcalgostyle{servers}}
\newcommand{\name}{\pcalgostyle{name}}
\newcommand{\context}{\pcalgostyle{context}}
\newcommand{\return}{\ensuremath{\mathbf{return}\ }}


\title{DRAFT Guess Limited Password Protected Secret Sharing Proposal}
\author{Rolfe Schmidt}


\begin{document}
\maketitle

\section{Overview}

This is a protocol for {\em guess-limited password-based secure value recovery}.
It allows clients to interact with servers to securely reconstruct a secret
using a password while providing protection against both offline and online
dictionary attacks - even in the event of server compromise. It protects against
online dictionary attacks through guess limiting: after a configured number of
failed reconstruction attempts, the secure value becomes unrecoverable.

\subsection{Outline}

This protocol is a variation of the PPSS protocol of \cite{jkkx} implemented
with a {\em usage limited} version of the standards track 2HashDH $\OPRF$ of
\cite{2hashdh} as specified in \cite{ietf-oprf} that can be used safely with
smaller curves like Ristretto255. 

After covering notation in section \ref{sec:notation} we present an augmentation
of the standards track $\OPRF$ of \cite{ietf-oprf} in section \ref{sec:oprf}
that has servers generate per-client $\OPRF$ keys, enforces strict usage limits
on these keys, and allows clients to rotate their keys to avoid running into
usage limits.

In section \ref{sec:ppss} we use this usage limited $\OPRF$ to construct a
secure PPSS. This protocol is close to that of \cite{jkkx}, but does not mandate
storage of masked shares on servers and eliminates the share commitment storage
on servers. We discuss ways to obtain robustness in section \ref{sec:robustness}. 

Importantly, we observe that if the underlying $\OPRF$ limits clients to
$\maxuses$ per key, then against a $(t,N)$ threshold scheme an attacker will be
limited to $\lfloor \frac{N}{t+1}\maxuses\rfloor$ password guess attempts before
the secret becomes unrecoverable. Thus our PPSS is {\em guess limited}. We also
show how keys can be deleted from the server to offer a form of forward security
in case of server compromise.


\section{Notation}
\label{sec:notation}
{\bf Algebraic objects.} This protocol will use a prime order cyclic group,
$\GG$ among those specified in \cite{ietf-oprf}. Since key use will be limited,
we can take $\GG$ to be Ristretto255. We
denote the order of $\GG$ by $q$ and thus denote the set of scalars for $\GG$ by
$\ZZ_{q}$. Group elements will be denoted by capital Latin letters, e.g. $A, B,
C, \ldots$. Scalars will be denoted by lower case Latin letters, e.g. $a, b,
c,\ldots$. $G$ denotes a public generator of $\GG$. Scalar multiplication will
usually be denoted without a symbol - $aG$ - but in places the infix operator
$*$ will be used for clarity, as in $\sk_{oprf}*G$. 

Secret sharing will be performed using polynomials over a finite field, $\FF$,
that is not related to the group $\GG$.

{\bf Domain separation.} Throughout the protocol we will use $\context$ to
denote a domain separation prefix unique to the application performing the
protocol.

{\bf Server and client state.} Each server will have state information captured
in the variable \serverstate. The public part of this state is available in the
variable \server. 

Similarly, each client will have persistent information captured in the variable
\clientstate, and the public part of this state will be accessible through the
variable \client.

{\bf Function parameters.} We will use a number of functions associated with
$\GG$ and $\FF$ which we consider as protocol parameters. In an instantiation of
the protocol the parameters will be identified in the \context string. The
function parameters are:
\begin{itemize}
    \item All parameters for the $\OPRF$ specified in \cite{ietf-oprf}
    \item $\HashToField: \bin^{*} \rightarrow \FF$
\end{itemize}



%%%%%%%%%%%%%%%%%%%%%%%%%%%%%%%%%%%%%%%%%%%%%%%%%%%%%%%%%%%%%%
%%
%%   OPRF
%%

\section{ The $\OPRF$}
\label{sec:oprf}
The PPSS protocol relies on the verifiable 2HashDH $\OPRF$ of \cite{2hashdh} as
specified in the IRTF draft standard \cite{ietf-oprf}. We describe the protocol
using the $\OPRF$ mode of the standard but in section \ref{sec:robustness} note 
situations where follow up calls to the $\VOPRF$ mode can be used to ensure 
robustness.

The IRTF standard specifies the following functions:
\begin{enumerate}
    \item $(\blind, \blindedElement) \leftarrow \Blind(\oprfinput)$: Used by a
    client to prepare the $\OPRF$ input to be sent to the server.
    \item $\evaluatedElement \leftarrow \BlindEvaluate(\sk,
    \blindedElement)$: Executed by the server to evaluate the $\OPRF$
    parameterized by the server-secret key $\sk$ on $\blindedElement$ to compute
    a blinded output.
    \item $v \leftarrow \Finalize(\oprfinput, \blind, \evaluatedElement,
    \blindedElement)$: Takes the values returned by $\BlindEvaluate$
    along with original input, and values returned by $\Blind$ to compute the 
    final $\prf$ value, $v$.
\end{enumerate}

\subsection{Usage Limited Evaluation}
The server for our protocol adds to these functions in two ways: it uses a
random per-client $\OPRF$ key stored in the dictionary $\serverstate.\oprfkeys$
and it enforces a usage limit. Each $\OPRF$ key can only be used a fixed number
of times. Setting and rotation of these keys is discussed in
\ref{sec:versioning}. This is done with the function $\BlindEvaluateForClient$:



\procedureblock[linenumbering]{$\BlindEvaluateForClient(\serverstate, \clientid,
\blindedElement)$}{ \usagecount \leftarrow \serverstate.\usagecount[\clientid]
\\
\serverstate.\usagecount[\clientid] \leftarrow \usagecount + 1 \\
\pcif \usagecount \geq \serverstate.\maxuses: \\
\t \pcreturn \perp \\
(\sk, \pk) \leftarrow \serverstate.\oprfkeys[\clientid]  \pccomment{The client
MAY obtain the \ensuremath{\pk} corresponding to \ensuremath{\sk} at
registration} \\
\pcreturn  \BlindEvaluate(\sk, \blindedElement) }



\subsection{$\OPRF$ Key Creation and Versioning}
\label{sec:versioning}
As noted in the previous section, $\OPRF$ keys are created per-client and each
key is strictly limited to a fixed number of uses. The usage limitation has two
useful purposes. First, since the security of the $\OPRF$ is based on the
one-more Diffie Hellman assumption, the security of a key used for $Q$ queries
is reduced by $\log(Q)/2$ bits (see, e.g.,
\href{https://www.ietf.org/id/draft-irtf-cfrg-voprf-21.html#section-7.2.3}{7.2.3
of the IRTF draft}). So, for example, by limiting key usage to no more than 16
queries we only lose 2 bits of security and can safely use a group like
Ristretto255. Second, this limit enforcement will be the basis of the guess
limiting in the PPSS described in section \ref{sec:ppss}.

Clients in our PPSS will need to reconstruct their secret an unlimited number of
times, though. To do this, upon successful reconstruction the client will create
a new version of their $\OPRF$ key. This new version will be constructed
with the function $\ServerCreateOPRFVersion$, which creates a new key pair,
stores it indexed by the client's identifier, clears the usage count, and
evaluates the $\OPRF$ with the new key on a blinded element:


\procedureblock[linenumbering]{$\ServerCreateOPRFVersion(\serverstate,
\clientid, \blindedElement)$}{ \sk \sample \ZZ_q \\
\pk \leftarrow kG \\
\serverstate.\oprfkeys[\clientid] \leftarrow (\sk, \pk)\\
\serverstate.\usagecount[\clientid] \leftarrow 0 \\
\evaluatedElement \leftarrow \BlindEvaluate(\sk, \blindedElement) \\
\pcreturn (\evaluatedElement, \pk) \pccomment{Return the public key in case \nizk\ proof is needed later} 
}

\subsection{A Note About $\POPRF$ Mode}
It is tempting to use the $\POPRF$ mode introduced in \cite{poprf} rather than
generating client specific keys. If usage limitation were not a requirement this
would have a clear advantage - the server state would be no more than one secret
$\OPRF$ scalar. However, once we introduce the need for key usage limits and
key rotation this advantage disappears. Usage limitation requires storage of
per-client state. Key rotation requires the use of a nonce or a counter for each
client, effectively requiring the same storage as the proposed per-client key
solution.

\section{A Guess Limited PPSS from the \OPRF}
\label{sec:ppss}
With these primitives in place we define the PPSS scheme with the functions
$\PPSSStore$ and $\PPSSRecover$. The idea is simple. To create a $(t,N)$
threshold PPSS to store a secret $s$ with $N$ servers we 
\begin{enumerate}
    \item Create a degree $t$ polynomial $f\in\FF[x]$ with $s$ as the leading
    coefficient, all other coefficients random.
    \item Create a share for each server: $s_i = f(x_i)$ where $x_i =
    \HashToField(\server_{i}.id)$
    \item Use the $\OPRF$ values to mask the shares: $m_i = s_i +
    \server_i.\OPRF(\clientid, pwd)$.
    \item Store the values $m_i$ somewhere reliable, but confidentiality is not
    important.
    \item To reconstruct, simply call $(t+1)$ or more servers to get their
    $\OPRF$ values and use these to unmask the shares: $s_i = m_i -
    \server_i.\OPRF(\clientid, pwd)$.
    \item These shares can now be used to reconstruct the secret $s$.
    \item Upon successful reconstruction the client can create new key versions
    on all servers, refresh their guess counts, and create new masked shares.
    All of this can be done without changing the password or master secret.
\end{enumerate}

In the following $\servers$ is a set of $N$ $\server$ objects, $\mathbf{e}$
is a dictionary that will store masked shares of a secret $s$, and $\mathbf{pks}$ 
is a dictionary that stores server $\OPRF$ public keys.
\procedureblock[linenumbering]{$\PPSSStore(\clientstate, \servers, t, pwd, s)$}{
r \concat K \leftarrow \hash(\context \concat ``keygen", s) \\
\forall i \in [0,t-1] : f_i \sample \FF \\
f_{t} \leftarrow \EncodeToField(s) \\
\pcfor \server \in \servers: \\
\t \oprfinput \leftarrow \context \concat \server.id \concat pwd \\
\t x \leftarrow \HashToField(\server.id) \\
\t y \leftarrow \sum_{i=0}^{t} f_i x^i \\
\t (\blind, \blindedElement) \leftarrow \Blind(\oprfinput) \\
\t (\evaluatedElement, \pk) \leftarrow
\server.\ServerCreateOPRFVersion(\clientstate.id, \blindedElement) \\
\t \rho \leftarrow \Finalize(\oprfinput, \blind, \evaluatedElement,\blindedElement) \\
\pccomment{ \ensuremath{\mathbf{e}} and \ensuremath{\mathbf{pks}} should be
stored somewhere reliable, but confidentiality is not needed} \\
\t \mathbf{s}[x] \leftarrow y \\
\t \clientstate.\mathbf{e}[x] \leftarrow y + \rho \\
\t \clientstate.\mathbf{pks}[\server.id] \leftarrow \pk \\
\clientstate.C \leftarrow \hash(\context\concat ``commitment", pwd, \clientstate.\mathbf{e}, \mathbf{s}, r) \\
\pcreturn K
}

\procedureblock[linenumbering]{$\PPSSRecover(\clientstate, \servers, t, pwd)$}{
\text{Choose } \mathcal{R} \subset \servers, |\mathcal{R}| > t \\
pairs \leftarrow \{\} \\
\pcfor \server \in \mathcal{R}: \\
\t \oprfinput \leftarrow \context \concat \server.id \concat pwd \\
\t x \leftarrow \HashToField(\server.id) \\
\t (\blind, \blindedElement) \leftarrow \Blind(\oprfinput) \\
\t (\evaluatedElement, \pk) \leftarrow
\server.\BlindEvaluateForClient(\clientstate.id, \blindedElement) \\
\t r \leftarrow \Finalize(\oprfinput, \blind, \evaluatedElement,\blindedElement) \\
\t y \leftarrow \clientstate.\mathbf{m}[x] - r \\
\t \mathbf{s}[x] \leftarrow y \\
\t pairs \leftarrow pairs \cup \{(x,y)\} \\
(f_{t}, \ldots, f_0) \leftarrow \pcalgostyle{Interpolate}_{\FF}(pairs) \\
s \leftarrow f_{t} \\
r \concat K \leftarrow \hash(\context \concat ``keygen", s) \\
C \leftarrow  \hash(\context\concat ``commitment", pwd, \clientstate.\mathbf{e}, \mathbf{s}, r) \\
\pcif C \neq \clientstate.C: \\
\t \pcreturn \perp \\
\pcelse \\
\t \PPSSStore(\clientstate, servers,t,pwd, f_{t}) \pccomment{store the secret again
to reset keys, counters, and shares} \\
\t \pcreturn K }

\subsection{Usage Limits on the $\OPRF$ lead to Guess Limits on the PPSS}
Now we can see how the usage limit we enforce on the $\OPRF$ naturally creates a
guess limit on the PPSS that provides protection against online dictionary
attacks. Consider the scenario where a client has constructed a $(t,N)$-sharing
scheme to protect a secret $s$ with password $pwd$ using $\PPSSStore$. Now an
attacker trying to guess the password and recover the secret faces the following
fact: each password guess requires using $t+1$ $\OPRF$ calls, and only
$\maxuses$ are possible on each of the $N$ servers. Thus the attacker has no
more than $\lfloor\frac{N}{t+1}\maxuses\rfloor$ guesses before the secret
becomes unrecoverable.

\subsection{Deleting Keys}
\label{sec:deleting}
A client can protect themselves from future server compromise by deleting keys
from the server. This can be done by simply calling $\ServerCreateOPRFVersion$
with arbitrary $\oprfinput$ and discarding the result. For the user to have
confidence that the keys were in fact deleted - now and during each $\PPSSStore$
call - server functions can be executed in an attested, confidential TEE.

Additionally, in the case that a TEE based server is being retired it can
produce an attested certificate of secret deletion.  If a client has confidence
that their secrets have, in fact, been deleted from a server then they know that
their $(t,N)$ threshold scheme has become a $(t,N-1)$ scheme and they can safely
add a new server.

\section{Robustness}
\label{sec:robustness}
Unlike \cite{jkkx} we do not store the commitment, $C$, and the masked shares, 
$\mathbf{e}$, on each server. Instead we will have clients store $\clientstate$,
which includes both both these values, in a reliable place as suggested in \cite{2hashdh}.
We then rely on subset testing or follow-up $\VOPRF$ calls to
detect incorrect servers.

The sole reason the server $\pk$ values are stored by the client in the protocol
above is to allow follow-up $\nizk$ proof verification if the $\VOPRF$ mode
is used for robustness. If only subset testing will be used (e.g. for small 
values of $N$) then these public keys do not need to be stored.

\section{Acknowledgements}  
We would like to thank Mark Johnson for helping to develop an earlier version of
this protocol and Trevor Perrin for important feedback and pointers to the
literature. We also thank Emma Dautermann, Vivian Fang, and Raluca Ada Popa for
discussion that led to significant design decisions and simplifications of this
protocol.

\bibliographystyle{alpha}
\bibliography{svr3}

\end{document}